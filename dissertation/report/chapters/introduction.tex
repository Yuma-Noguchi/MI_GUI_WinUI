\chapter{Introduction}

\section{Context and Motivation}
MotionInput is an innovative accessibility software that enables users with physical disabilities to interact with computers and games through alternative input methods. While the core functionality of MotionInput has proven valuable, the need for an intuitive and user-friendly configuration interface has become increasingly apparent.

\section{Problem Statement}
The existing configuration process for MotionInput requires manual editing of configuration files, which presents a significant barrier for users, especially those with limited technical expertise. This complexity can deter potential users and limit the software's accessibility benefits.

\section{Project Objectives}
The primary objectives of this project are:
\begin{itemize}
    \item To develop a graphical user interface for managing MotionInput configurations
    \item To simplify the process of creating and modifying visual profiles
    \item To provide real-time preview and testing capabilities
    \item To ensure the interface itself adheres to accessibility guidelines
    \item To integrate seamlessly with the existing MotionInput ecosystem
\end{itemize}

\section{Technical Approach}
The project utilizes the WinUI 3 framework to create a modern, native Windows application. This choice ensures optimal performance, system integration, and accessibility features while maintaining consistency with Windows' design language.

\section{Contributions}
The key contributions of this project include:
\begin{itemize}
    \item A user-friendly interface for managing MotionInput configurations
    \item An intuitive visual profile editor
    \item Real-time configuration preview capabilities
    \item Enhanced profile management features
    \item Comprehensive documentation and user guides
\end{itemize}

\section{Dissertation Structure}
The remainder of this dissertation is organized as follows:
\begin{itemize}
    \item Chapter 2 reviews relevant background literature and existing solutions
    \item Chapter 3 details the requirements gathering and analysis process
    \item Chapter 4 describes the implementation of the Configuration GUI
    \item Chapter 5 presents testing and evaluation results
    \item Chapter 6 concludes with reflections and future work recommendations
\end{itemize}
