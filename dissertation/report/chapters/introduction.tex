\chapter{Introduction}

\section{Context and Motivation}
Digital gaming has evolved from a niche hobby to a mainstream cultural phenomenon, with over 3 billion gamers worldwide as of 2023 \cite{newzoo2023}. However, this rapid growth has not been equally accessible to all. An estimated 46 million gamers in the United States alone have disabilities that affect their gaming experience \cite{ablegamers2023}, with physical input limitations representing one of the most significant barriers to participation.

MotionInput represents a pioneering solution in this accessibility landscape. Developed at University College London, this software enables users with physical disabilities to interact with computers and video games through alternative input methods, including camera-based motion detection, voice commands, and adapted controllers. By translating natural body movements and gestures into standard input commands, MotionInput creates a bridge between users' physical capabilities and digital experiences.

While the core functionality of MotionInput has demonstrated significant value for users with diverse abilities, the configuration process has remained a substantial technical hurdle. As the system has evolved to support more sophisticated interaction patterns and game-specific profiles, the complexity of the underlying configuration system has increased proportionally, creating an accessibility paradox: software designed to improve accessibility has itself become less accessible due to configuration complexity.

\section{Problem Statement}
The existing configuration system for MotionInput relies on manual editing of JSON configuration files. This approach requires users to navigate complex nested JSON structures with strict syntax requirements and manually define input-to-action mappings through technical parameter specifications. Users must memorize specific parameter names and allowed values while troubleshooting configuration errors through cryptic error messages. Additionally, they must manage multiple configuration files across different applications and games, creating a significant cognitive burden.

This technical barrier severely limits the software's effectiveness across several dimensions. The complexity deters potential users, particularly those with limited computing experience, effectively restricting adoption to those with technical backgrounds. Manual JSON editing inevitably leads to frequent syntax and logical errors, creating frustration and wasted time. The disconnect between technical configuration and visual outcomes produces a confusing user experience that contradicts the software's accessibility mission. As a result, many users avoid creating custom configurations altogether, limiting the potential benefits of the software, while those who attempt customization often require technical assistance from developers or community members.

These issues fundamentally undermine the core mission of MotionInput: to improve digital accessibility. A solution that addresses these configuration barriers is essential to realizing the full potential of the technology and reaching a broader audience of users who could benefit from its capabilities.

\section{Project Objectives}
This project aims to transform the MotionInput configuration experience through a purpose-built graphical user interface that replaces manual JSON editing with intuitive visual tools. The primary goal is to create an interface that maintains the full functionality and flexibility of the underlying system while eliminating the technical barriers that currently limit its accessibility.

At the heart of this transformation is the development of a visual profile management system that allows users to create, edit, and manage input profiles without any knowledge of JSON structure or syntax. This system will provide immediate visual feedback through real-time preview capabilities, helping users understand the relationship between configuration changes and resulting behaviors.

To enhance the visual distinctiveness of different profiles, the system will integrate AI-generated visual elements leveraging Stable Diffusion technology. This innovation will allow users to automatically generate context-appropriate icons and visual cues based on simple text descriptions, eliminating the need for manual graphic design work.

Throughout development, accessibility will remain a central focus, ensuring the configuration interface itself follows WCAG 2.1 accessibility guidelines. This adherence to standards will include screen reader compatibility, keyboard navigation support, and appropriate color contrast, making the configuration interface accessible to users with diverse abilities.

Technical integration objectives include ensuring seamless compatibility with the existing MotionInput ecosystem and optimizing performance to deliver responsive interactions even on modest hardware configurations. Success will be measured through comprehensive usability testing, error rate reduction, configuration time improvements, and user satisfaction metrics.

\section{Technical Approach}
The project employs a comprehensive technical approach centered on modern Windows development technologies and AI integration. The WinUI 3 framework serves as the foundation for the user interface, selected for its modern UI capabilities, performance characteristics, and native Windows integration. This framework provides fluid animations, responsive layouts, and comprehensive accessibility features essential to the project's success.

The implementation follows the Model-View-ViewModel (MVVM) architectural pattern, ensuring clear separation of concerns between the data model, business logic, and user interface elements. This pattern not only improves code organization and maintainability but also facilitates comprehensive testing of individual components without dependencies on other parts of the system.

A key innovation in the project is the integration of AI capabilities through Stable Diffusion, implemented using the ONNX Runtime. This integration enables intelligent, context-aware visual element generation, significantly reducing the barrier to creating visually distinct and meaningful profiles. The service-oriented design approach modularizes functionality into discrete, reusable services that handle specific aspects of the application's behavior, from file management to configuration validation to AI processing.

Accessibility considerations permeate every aspect of the development process, with implementations of screen reader support, keyboard navigation, high contrast compatibility, and other essential accessibility features. This accessibility-first development approach ensures that the configuration tool itself does not create new barriers for the users it aims to serve.

This technical foundation enables the creation of a sophisticated yet intuitive interface while maintaining the performance and reliability required for a seamless user experience across a wide range of hardware configurations.

\section{Contributions}
This project makes several significant contributions to the field of accessible gaming and human-computer interaction. The introduction of a visual approach to accessibility configuration represents a novel configuration paradigm that challenges the text-based systems that have dominated accessibility tools. By demonstrating that complex configuration tasks can be accomplished through intuitive visual interfaces, this project establishes a model for future accessibility tool development.

The integration of generative AI for automatic visual asset creation demonstrates a practical application of emerging AI technologies in accessibility tools. This approach not only simplifies the user experience but also showcases how AI can be leveraged to reduce technical barriers in accessibility contexts. The project establishes reusable accessibility design patterns for creating inclusive configuration interfaces, contributing valuable insights to the broader field of accessible UI design.

Through careful optimization and testing, the project develops techniques for maintaining responsiveness in complex, data-driven interfaces, particularly important for accessibility applications where performance issues can create additional barriers. Finally, by providing a fully documented, extensible codebase released as open source, the project creates a foundation for future accessibility research and development that can be built upon by the broader community.

These contributions extend beyond the immediate project scope, offering insights and techniques applicable to the broader field of accessible technology development and establishing pathways for continued innovation in this critical area.

\section{Dissertation Structure}
The remainder of this dissertation is organized as follows:

\begin{itemize}
    \item \textbf{Chapter 2: Background and Literature Review} examines the theoretical foundations and existing work in gaming accessibility, configuration interfaces, and AI integration in user interfaces. This chapter provides essential context for understanding the project's significance and approach.
    
    \item \textbf{Chapter 3: Requirements Analysis and Specification} details the systematic process of gathering, analyzing, and specifying requirements, including functional, non-functional, and accessibility requirements.
    
    \item \textbf{Chapter 4: Implementation} documents the technical implementation of the Configuration GUI, covering architecture, interface design, AI integration, and system optimization.
    
    \item \textbf{Chapter 5: Testing and Evaluation} presents the testing methodology, results, and user feedback, providing critical assessment of the project's success against its objectives.
    
    \item \textbf{Chapter 6: Conclusion} summarizes key achievements, critically evaluates the project, identifies limitations, and suggests directions for future work.
\end{itemize}