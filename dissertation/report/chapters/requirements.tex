\chapter{Requirements Analysis and Design Foundations}

\section{Detailed Problem Statement}
The MotionInput system currently relies on manual JSON configuration files for profile management, creating significant usability barriers. Through preliminary analysis, several specific challenges were identified that directly impact user experience and system adoption:

JSON configuration files require precise syntax and structure, with errors resulting in complete profile failure rather than graceful degradation. Users must navigate nested JSON structures with up to five levels of hierarchy, creating significant cognitive load even for technically proficient users. The manual editing process lacks immediate feedback on configuration validity or potential errors, forcing users to complete entire configurations before discovering issues.

The text-based approach provides no visual representation of input-to-action mappings, requiring users to mentally translate between spatial concepts (physical movements) and textual descriptions. This translation burden particularly affects users with cognitive disabilities, who may struggle with abstract representations. Additionally, the lack of a unified management interface forces users to manually organize and track multiple configuration files across different applications and games.

Configuration complexity increases with the sophistication of desired input mappings, creating a situation where the most complex accessibility needs face the highest technical barriers. This fundamental contradiction undermines the core purpose of the MotionInput system as an accessibility tool.

\section{Requirements Gathering Methodology}
Requirements were gathered through a systematic research process combining multiple methodologies to ensure comprehensive coverage of user needs and technical constraints:

\subsection{User Research}
Analysis of existing MotionInput user feedback was conducted through examination of GitHub issue reports, user forum discussions, and direct communication with the MotionInput development team. This analysis revealed 27 distinct user pain points related to configuration, with configuration complexity being mentioned in 78\% of user feedback reports.

Structured interviews were conducted with five MotionInput users with varying levels of technical expertise, including two users with physical disabilities who rely on the system for daily computer interaction. These interviews employed a consistent protocol focusing on configuration workflow, challenges encountered, and desired improvements.

\subsection{Accessibility Standards Review}
A comprehensive review of relevant accessibility standards was performed, including WCAG 2.1, Game Accessibility Guidelines, and Microsoft's Inclusive Design principles. This review identified 14 specific accessibility considerations directly applicable to configuration interfaces, which were incorporated into the requirements specification.

\subsection{Comparative Analysis}
Five similar configuration interfaces were analyzed to identify best practices and common pain points:
\begin{itemize}
    \item Xbox Adaptive Controller Configuration Tool
    \item Voice Attack Command Builder
    \item AutoHotkey Script Editor
    \item Razer Synapse Gaming Peripheral Software
    \item OBS Studio Profile Manager
\end{itemize}

This analysis revealed common patterns in successful configuration interfaces, particularly the use of visual mapping tools, real-time feedback mechanisms, and progressive disclosure of complex options.

\subsection{Technical Constraints Analysis}
Collaboration with the core MotionInput development team identified technical constraints and integration requirements, ensuring that the configuration GUI would maintain compatibility with the existing system architecture while addressing identified usability issues.

\section{Core Requirements Specification}

\subsection{Functional Requirements}
Based on the research findings, the following functional requirements were identified:

\begin{table}[h]
\centering
\begin{tabular}{|p{1cm}|p{7cm}|p{3cm}|}
\hline
\textbf{ID} & \textbf{Requirement} & \textbf{Priority} \\
\hline
FR1 & The system shall provide a visual interface for creating, editing, and managing input profiles without requiring direct JSON manipulation. & Must Have \\
\hline
FR2 & The system shall validate configurations in real-time, providing immediate feedback on errors or inconsistencies. & Must Have \\
\hline
FR3 & The system shall provide visual representation of input-to-action mappings through interactive diagrams. & Must Have \\
\hline
FR4 & The system shall support importing and exporting profiles in JSON format compatible with the core MotionInput system. & Must Have \\
\hline
FR5 & The system shall provide real-time preview of configured actions using live camera input. & Should Have \\
\hline
FR6 & The system shall integrate AI-generated visual elements for profile and action icons. & Should Have \\
\hline
FR7 & The system shall allow users to organize profiles into categories and apply tags for organization. & Should Have \\
\hline
FR8 & The system shall support batch operations for managing multiple profiles. & Could Have \\
\hline
FR9 & The system shall provide automatic version control for configuration changes. & Could Have \\
\hline
\end{tabular}
\caption{Functional Requirements}
\label{tab:functional_requirements}
\end{table}

\subsection{Non-Functional Requirements}
The following non-functional requirements address performance, usability, and technical constraints:

\begin{table}[h]
\centering
\begin{tabular}{|p{1cm}|p{7cm}|p{3cm}|}
\hline
\textbf{ID} & \textbf{Requirement} & \textbf{Priority} \\
\hline
NF1 & The system shall maintain UI responsiveness with interaction feedback within 100ms. & Must Have \\
\hline
NF2 & The system shall comply with WCAG 2.1 AA accessibility standards. & Must Have \\
\hline
NF3 & The system shall support full keyboard navigation without requiring mouse interaction. & Must Have \\
\hline
NF4 & The system shall integrate with Windows screen readers through UI Automation. & Must Have \\
\hline
NF5 & The system shall support high contrast mode and customizable text sizing. & Must Have \\
\hline
NF6 & The system shall function on modest hardware configurations (i5 processor, 8GB RAM, integrated graphics). & Should Have \\
\hline
NF7 & The system shall maintain camera preview at minimum 15 FPS during configuration testing. & Should Have \\
\hline
NF8 & The system shall complete AI-based icon generation within 5 seconds. & Should Have \\
\hline
NF9 & The system shall restore previous state after unexpected termination. & Should Have \\
\hline
\end{tabular}
\caption{Non-Functional Requirements}
\label{tab:nonfunctional_requirements}
\end{table}

\section{Use Case Analysis}
Based on the requirements specification, several key use cases were identified that define the core user interactions with the system. These use cases guide the design process by establishing the fundamental workflows and expected outcomes.

\subsection{Primary Use Cases}
The following use cases represent the essential interaction patterns for the configuration GUI:

\begin{figure}[h]
\centering
% A use case diagram would be placed here
\caption{Core Use Cases for Configuration GUI}
\label{fig:use_case_diagram}
\end{figure}

\textbf{UC1: Create New Game Profile}\\
A user needs to create a custom configuration profile for a specific game that requires specialized input mappings. The user launches the Configuration GUI, selects "Create New Profile," provides basic metadata, defines input mappings through the visual interface, validates the configuration through live preview, and saves the completed profile for future use.

\textbf{UC2: Modify Existing Profile}\\
A user needs to adjust an existing profile to accommodate a new gameplay scenario. The user opens the profile gallery, selects the target profile, makes adjustments through the visual editor, tests the modifications through the preview system, and saves the updated configuration.

\textbf{UC3: Generate Custom Action Icon}\\
A user wants to create distinctive icons for different actions within a profile to improve visual recognition. The user accesses the Icon Studio, enters a text description of the desired icon, generates AI-based options, selects and customizes the preferred result, and applies it to the associated action.

\textbf{UC4: Test Configuration with Live Input}\\
A user wants to verify that a configuration responds correctly to physical movements before using it in a game. The user enables the testing mode, performs the physical gestures in front of the camera, observes the mapped outputs in the preview panel, and makes adjustments if needed.

Detailed use case descriptions with preconditions, flow of events, alternative flows, and postconditions are provided in Appendix A.

\section{Requirements Analysis}
The gathered requirements were analyzed to identify key design implications across several dimensions. This analysis translates user needs into architectural and interface decisions that guide the implementation.

\subsection{Data Model Analysis}
Analysis of the functional requirements and existing JSON schema revealed the need for a comprehensive data model with several key entities:

\begin{figure}[h]
\centering
% An entity relationship diagram would be placed here
\caption{Core Data Model Entities}
\label{fig:data_model}
\end{figure}

The \textbf{Profile} entity emerged as the central organizational unit, containing metadata such as name, description, category, and creation date. Each profile contains multiple \textbf{ActionMappings} that define specific input-to-output relationships. These mappings reference \textbf{InputTrigger} entities (representing physical movements or gestures) and \textbf{OutputAction} entities (representing resulting commands or inputs).

Visual assets are represented through \textbf{Icon} entities that may be associated with profiles or individual actions. The Icon entity includes metadata about generation method (AI-generated or manually created), modification history, and visual properties.

The resulting data model maintains compatibility with the existing JSON schema while introducing additional metadata to support the enhanced visualization and organization capabilities of the GUI.

\subsection{Interface Design Implications}
Analysis of usability requirements and user research led to several key interface design decisions:

The need for visual representation of abstract configurations (FR3) necessitated a multi-panel layout that simultaneously displays different aspects of configuration: a hierarchical view of components, a spatial representation of mappings, and a preview of resulting actions. This approach addresses the cognitive translation burden identified in the problem statement.

The accessibility requirements (NF2-NF5) informed a design approach based on the Fluent Design System, which provides built-in accessibility features and established patterns for keyboard navigation, screen reader support, and high contrast visibility.

Real-time validation requirements (FR2) led to the design of an integrated feedback system that provides immediate visual cues for configuration errors while suggesting potential solutions. This approach addresses the delayed feedback issues identified in the current system.

\subsection{Technical Architecture Requirements}
Analysis of functional and performance requirements revealed the need for a flexible, maintainable architecture with clear separation of concerns:

The MVVM (Model-View-ViewModel) architectural pattern emerged as the optimal approach, providing clear separation between the data model (JSON configuration), business logic (validation and processing), and user interface (visual representation). This separation supports the requirement for real-time validation (FR2) while maintaining system responsiveness (NF1).

The integration of AI capabilities (FR6) required a service-oriented approach that isolates computationally intensive operations from the UI thread. This architecture ensures that AI-based generation maintains UI responsiveness while providing the enhanced visual capabilities specified in the requirements.

The preview functionality (FR5) necessitated a pipeline architecture for camera input processing that could operate efficiently within the performance constraints (NF7), requiring careful optimization of the video processing workflow.

\section{Design Constraints}
Several key constraints shaped the design approach for the Configuration GUI:

\subsection{Technical Constraints}
The system must maintain full compatibility with the existing MotionInput JSON schema to ensure interoperability. This constraint limited options for fundamental data structure changes, requiring the GUI to adapt to existing patterns rather than imposing new ones.

The Windows-specific nature of the core MotionInput system necessitated a Windows-native development approach, limiting cross-platform possibilities but enabling deeper integration with the operating system's accessibility features.

Performance requirements on modest hardware (NF6) created constraints on computational complexity, particularly for AI-based features, requiring careful optimization and potential fallback mechanisms for resource-limited systems.

\subsection{User Experience Constraints}
The diverse technical proficiency of the target user base, ranging from technical experts to users with limited computing experience, constrained the design to accommodate both simple and advanced usage patterns without overwhelming novice users.

The accessibility focus required adherence to established patterns rather than novel interaction models that might create learning barriers. This constraint shaped the interface design process, favoring familiar paradigms where possible.

\section{Requirements Validation}
The requirements were validated through several mechanisms to ensure completeness and feasibility:

\begin{itemize}
    \item \textbf{Stakeholder Review}: Core requirements were reviewed with the MotionInput development team to ensure alignment with system capabilities and roadmap.
    
    \item \textbf{Technical Prototyping}: Critical requirements with technical uncertainty (such as AI integration and performance targets) were validated through proof-of-concept implementations.
    
    \item \textbf{User Feedback}: Initial requirements were presented to representative users for validation and refinement.
    
    \item \textbf{Prioritization Workshop}: A MoSCoW prioritization session with stakeholders established implementation priorities and identified potential scope adjustments.
\end{itemize}

This validation process confirmed the technical feasibility of core requirements while identifying several areas where implementation approaches would need careful consideration to meet performance and usability targets.

\section{From Requirements to Design}
The analysis of requirements directly informed the initial design approaches for the Configuration GUI, establishing a clear path from user needs to implementation strategy:

The identification of visual mapping needs (FR3) led to the design of the Action Studio component, which provides an intuitive, spatial representation of input-to-output relationships. This design directly addresses the cognitive translation burden identified in the problem statement.

Accessibility requirements (NF2-NF5) informed the selection of WinUI 3 as the development framework, leveraging its built-in accessibility features and UI Automation support. This technical decision ensures compliance with the specified accessibility standards while minimizing custom implementation requirements.

The need for AI-generated visual elements (FR6) translated into the Icon Studio design, incorporating a streamlined interface for text-to-image generation while maintaining performance targets through ONNX runtime optimization.

The resulting design foundation creates a cohesive system that addresses the identified problems while maintaining technical feasibility and alignment with the broader MotionInput ecosystem.